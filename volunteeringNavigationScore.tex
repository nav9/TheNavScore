\documentclass[conference]{IEEEtran}
\IEEEoverridecommandlockouts
% The preceding line is only needed to identify funding in the first footnote. If that is unneeded, please comment it out.
\usepackage{cite}
\usepackage{amsmath,amssymb,amsfonts}
\usepackage{algorithmic}
\usepackage{graphicx}
\usepackage{textcomp}
\usepackage{xcolor}
\usepackage{csquotes}
\def\BibTeX{{\rm B\kern-.05em{\sc i\kern-.025em b}\kern-.08em
    T\kern-.1667em\lower.7ex\hbox{E}\kern-.125emX}}
\begin{document}

\title{The Nav Score: A Simple Measure to Evaluate and Guide Volunteering Teams\\
{\footnotesize }
\thanks{}
}

\author{
\IEEEauthorblockN{Navin K Ipe}
\IEEEauthorblockA{\textit{A humble volunteer for social causes} \\
\textit{1\textsuperscript{st} September, 2015}\\
Bengaluru, India \\
navinipe@gmail.com}
}

\maketitle

\begin{abstract}
Corporate Social Responsibility teams, NGO's and individual volunteers often find themselves lacking in knowledge and direction on how to help the world. Many are even coerced into doing social service, leading to the formation of superficial teams that unknowingly cause irreversible damage to the causes they were attempting to positively influence. This paper intends to provide companies, NGO's and individuals a measurable guideline to help them evaluate and re-evaluate their strategy to \enquote{be the change} they want to see in the world.

\end{abstract}

\begin{IEEEkeywords}
CSR, volunteering, NGO, impact, metric
\end{IEEEkeywords}

\section{Introduction}
Organizing a volunteering activity involves an extremely detailed process that takes into account the need to be fulfilled, consulting with experts in the field who are already dealing with solving the problem, finding individuals who are truly enthusiastic about solving the problem, performing a detailed analysis on finding the best ways to solve the problem in a way that is sustainable, obtaining a concensus from everyone involved, having a mechanism to manage the team and handle disruptive volunteers, executing the plan and checking if the effort truly helped improve the situation. It is only after these core practices are put in place that volunteering teams add a few fun-activities, to help attract more helping-hands. Over the years, people have observed the fun-activities and have come to believe that those activities are the core of a volunteering activity. Individuals, NGO's and corporate volunteering teams spend a significant amount of time and money organizing for celebrations and short-duration activities \cite{orphanageVisits1} in the name of CSR and employee volunteering, without taking into account the core problems of the social problem they intend to help solve.
Current measures of the impact of CSR focus on volunteer man-hours\cite{employeeVolunteerism}, employee engagement \cite{impactOnEmployees}\cite{intangibles} or the metrics of how efficiently resources were used to deliver impact\cite{classificationOfSocialImpact}\cite{businessCaseCSR}. These measures perform a high-level analysis which do not always take into account the ground-level realities of how prepared the teams are and whether their work is done with a conscience \cite{truthAboutCSR} or they are excellent measures but far too detailed and complicated \cite{detailedMeasure}.

\section{Existing issues}
Misguided kindness \cite{misguidedKindness}, as superficial volunteering can aptly be termed, has become a social menace in its own right. It has given birth to a culture where volunteers are encouraged to organize little volunteering activities that last for a few hours, do not actually help the beneficiaries, involves taking a few photographs and proclamations of having being part of a \enquote{noble cause}. Such activities have a profoundly negative impact on all fields of volunteering, but most prominetly on children in orphanages, people in destitute homes and people in old-age homes. Studies have shown \cite{orphanageVisits2} that short-term orphanage visits lowers a child's sense of privacy in their own home \cite{guidelinesOnAlternativeCare}, can harm a child's development and emotional well-being, as institutionalized children tend to manifest the same indiscriminate affection toward volunteers and the attachment is broken when the volunteer leaves and never returns. Evidence from studies show that repeated disruptions in attachment are extremely disrupting for young children. From a personal experience, a child in a disabled home mentioned how a priest from a neighbouring church encouraged children to help out at the disabled home for at least fifteen minutes. The children came to help, looked at their watch and discussed amongst themselves (right in front of the disabled children) \enquote{Thirteen minutes have elapsed. By the time we walk to the gate it'll be fifteen minutes}. The head of an orphanage in India mentioned how girls in the orphanage disliked having (well-meaning) visitors request the girls to sing a song or recite a poem for them. The head of an orphanage for AIDS affected children mentioned how the help they really appreciate from people would be medicines or protien shakes for the children, but what visitors bring are cakes, soft-drinks and toys (even after the volunteers are specifically requested not to gift those) which pile up unused in the orphanage. Caretakers of destitute homes and old age homes have also expressed their chagrin at having corporate volunteers who visit every weekend, pass time and go away. People at these homes know the volunteers do not really care.
Whether it's helping people, planting saplings, donating blood, educating children, saving water, reducing plastic, saving electricity or any other activity, volunteers need to learn how to solicit knowledge and plan for the long-term before embarking on their mission to create a better world. Not doing the necessary research does more harm than good, and does not come under the ambit of \enquote{noble cause}.

\section{Proposed guideline}
The Volunteering Navigation Test, abbreviated as the Nav Test, was designed to help CSR groups, volunteers and NGO's have a simple, measurable score that gives them an estimate of whether they are headed in the right direction and what steps they need to take to get their volunteering right. The scoring parameters were evaluated by a Social Entrepreneurship professor from the Copenhagen Business School and by the head of a leading blood donation NGO in Bengaluru. Corrections were made to the test parameters and is currently frozen as a score reference for any volunteering team that would like to improve and be true to their cause in meaningfully help make the world a better place.
The scoring system of the Nav test is not a critique, but a guideline which can help teams identify:
\begin{itemize}
\item The right direction when volunteering.
\item What the essential facets of a volunteering activity are.
\item Whether the activity is actually helping the beneficiaries.
\item Whether the team has checks and balances to keep itself stable and knowledgeable.
\end{itemize}

\section{The Nav Test}
The Nav Test begins with a score of zero. One point is added for every \enquote{Yes} to a bulleted point. No points are added or subtracted for a \enquote{No}. However, if a bulleted point is accompanied by a condition that gives a negative scoring, then the corresponding number of points mentioned are to be subtracted.
\begin{itemize}
\item One plus point if the team has a long-term commitment to improving a certain area of society at least every month (two minus points if they jump from one activity/area to another in less than two years).
\item One plus point if the team maintains data to measure if their work really helped the beneficiaries and analyze if it could be done better by finding a way to solicit honest feedback (One minus point if the only data maintained is a bunch of photos, number of items donated and/or the number of volunteer man-hours. Two minus points if no data is maintained).
\item One plus point if the team consults with experts in the field, for knowledge (One additional plus point if the team is the expert and is building knowledge that is freely shared).
\item One plus point if the team takes pains to actually understand and solve the cause of the social problem, analyzing it from the point of view of the various factors that contributed to the problem.
\item One plus point if when a volunteer joins the team, the team informs the volunteer of their past, identifies the volunteer's skills, interests and comfort-level to ensure that the volunteer contributes their best. (One additional plus point if they educate the volunteer further)
\item One plus point if the team has a mechanism to handle disruptive volunteers \cite{disruptiveVolunteers}.
\end{itemize}

\section{Results}
An average team would score four points. A very good team would score six points. An exceptional team would score eight points.
It would not be surprising if certain team scores figure in negative numbers. It's good if it does, since now they would know where to improve. Acts of philanthropy are driven by vanity, coercion or a genuine desire to help. The Nav Test or any other guideline for volunteers can only help the latter two cases. Before deciding to join or fund a volunteering team or NGO, check if they have mechanisms in place to responsibly measure the work they do. Any responsible team will ask the question of whether they are helping society in a sustainable manner. They will be on the lookout for a measure like the Nav Test or will create their own measure where the purpose of the measure is not to show impressive data charts, but the purpose serves to check if the beneficiaries have been helped and whether their situation can be improved to a stage that they would become self-dependent. Sadly, many CSR activities today do not account for the \enquote{R} part of CSR, and attempt at being independent heroes instead of soliciting expert help from people who have been working in the social sector for decades and are aware of problems at the grassroot level. 

\section{Conclusions}
Truly helping society is an arduous, monotonous task that requires many years of dedicated effort. Organizing short-term fun-activities that pretend to help society, does more harm than good. It is imperative that teams organizing volunteering efforts put in more effort to educate volunteers before they venture into the field. The social sector has many dedicated experts who are more than willing to help with advice and pointers to the most pressing problems in the field. The Nav Test is just one small step that hopes to encourage people to take that giant leap in changing the volunteering culture from being a superficial activity, to becoming a more dedicated, coordinated effort that truly helps make the world a better place.


\begin{thebibliography}{00}
\bibitem{impactOnEmployees} Alin Stancu, Georgiana Florentina Grigore, Mihai Ioan Rosca (2011), ``The Impact of Corporate Social Responsibility on Employees''. International Conference on Information and Finance IPEDR vol.21.
\bibitem{intangibles} Cristiana Parisi, Kai N. Hockerts (2013), ``Managerial mindsets and performance measurement systems of CSR‐related intangibles''. Measuring Business Excellence, Vol. 12 Issue: 2, pp.51-67.
\bibitem{misguidedKindness} Joanne Doyle (2010), ``Misguided Kindness. Making the right decisions for children in emergencies''. Save the children, UK. 2010.
\bibitem{classificationOfSocialImpact}  Karen Maas, Kellie Liket (2011), ``Social Impact Measurement: Classification of Methods''.  In: Burritt R., Schaltegger S., Bennett M., Pohjola T., Csutora M. (eds) Environmental Management Accounting and Supply Chain Management. Eco-Efficiency in Industry and Science, vol 27. Springer, Dordrecht.
\bibitem{truthAboutCSR} Kasturi Rangan, Lisa Chase, Sohel Karim (2015). ``The truth about CSR''. Oceania article HBR social responsibility 2015.
\bibitem{orphanageVisits2}  Kathie Carpenter (2015), ``Childhood studies and orphanage tourism in Cambodia'' Annals of Tourism Research, Volume 55, November 2015, Pages 15-27.
\bibitem{disruptiveVolunteers} McCurley, Steve, Sue Vineyard (1998), ``Handling problem volunteers''. Heritage Arts Publishing, Downers Grove, II. pp. 41-52.
\bibitem{businessCaseCSR}  ManuelaWeber (2008), ``The business case for corporate social responsibility: A company-level measurement approach for CSR''.  Volume 26, Issue 4, August 2008, Pages 247-261.
\bibitem{detailedMeasure} Marc J. Epstein (2008), ``Making Sustainability Work Best Practices in Managing and Measuring Corporate Social, Environmental and Economic Impacts''.  London: Routledge.
\bibitem{employeeVolunteerism} Susan M. Houghton, Joan T. A. Gabel, David W. Williams (2008), ``Connecting the Two Faces of CSR: Does Employee Volunteerism Improve Compliance?''. D.W. J Bus Ethics (2009) 87: 477. https://doi.org/10.1007/s10551-008-9954-2.
\bibitem{orphanageVisits1}  Tess Guiney, Mary Mostafanezhad (2014), ``The political economy of orphanage tourism in Cambodia'' Tourist studies. December 30, 2014.
\bibitem{guidelinesOnAlternativeCare} UN General Assembly (2011), ``Guidelines for the Alternative Care of Children''. Sixty-fourth session Agenda item 64. Resolution adopted by the General Assembly [on the report of the Third Committee (A/64/434)].


\end{thebibliography}

\end{document}
